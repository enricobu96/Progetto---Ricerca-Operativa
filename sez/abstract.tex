\newpage
\section{Abstract}
Il problema che segue consiste in un classico problema di mix ottimo di produzione. Attraverso la modellazione matematica si vuole decidere il piano di produzione di una ditta di strumenti musicali, nello specifico chitarre e bassi, al fine di massimizzare il guadagno complessivo annuale dato dalla vendita di suddetti strumenti. \\ 
In un primo momento verrà definito il testo del problema, comprendente tutti i dati riguardanti i costi di produzione, la richiesta da parte del mercato, eventuali condizioni di cui l'azienda deve tener conto (\textbf{e.g.} la quantità massima di manodopera e relativo costo).
Successivamente verrà creato un modello per il problema tramite la \textit{programmazione lineare} che, una volta trasposto nel linguaggio di programmazione \textit{AMPL}, porterà alla determinazione del guadagno complessivo massimo per l'azienda e quindi al piano di produzione che la ditta dovrà seguire per ottenere tale guadagno.