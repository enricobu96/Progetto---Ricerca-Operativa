\newpage
\section{Risoluzione con il software AMPL}
\subsection{Panoramica}
Per risolvere il problema è stato utilizzato il software \textit{AMPL}, con il solver \textit{cplex} alla sua versione 12.8, il tutto nell'ambiente di programmazione integrato \textit{AMPLIde}. \\
Sono stati realizzati cinque files, i quali sono:
\begin{itemize} 
\item \textbf{candb.mod}: contiene il modello del problema; in esso sono definiti gli insiemi, i parametri, le variabili decisionali, la funzione obiettivo e i vincoli;
\item \textbf{candb.dat}: contiene i dati del problema; in esso vengono inizializzati tutti gli insiemi e i parametri. Lo scopo di avere i dati separati dal modello è la versatilità: si possono infatti modificare tutti i dati del problema senza andare a toccare il modello;
\item \textbf{candb2.dat}: analogo al primo file \textbf{.dat} ma contenente un diverso set di dati;
\item \textbf{candb.run}: script creato per facilitare il caricamento di modello e dati nel software e permettere una visualizzazione ordinata dei dati di output;
\item \textbf{candb2.run}: analogo al primo file \textbf{.run} per l'automazione della risoluzione del problema con il secondo set di dati.
\end{itemize}

\subsection{Risultati}
Dopo la modellazione in AMPL, è stato eseguito il solver \textit{cplex} per il calcolo del profitto massimo dell'azienda e la miglior distribuzione della produzione dei vari strumenti tra gli stabilimenti. Di seguito è riportato l'output dell'esecuzione.
\begin{verbatim}
Risultati dell'esecuzione: 
 
================================ 
VISUALIZZAZIONE PER STABILIMENTI 
================================ 
Stabilimento A 
Strumento 1 : 849
Strumento 2 : 5093
Strumento 3 : 2082
Strumento 4 : 0
Strumento 5 : 1
Strumento 6 : 0
Stabilimento B 
Strumento 1 : 0
Strumento 2 : 1
Strumento 3 : 2232
Strumento 4 : 0
Strumento 5 : 0
Strumento 6 : 92
Stabilimento C 
Strumento 1 : 0
Strumento 2 : 0
Strumento 3 : 0
Strumento 4 : 0
Strumento 5 : 10912
Strumento 6 : 0

============================= 
VISUALIZZAZIONE PER STRUMENTI 
============================= 
Strumento 1 
Stabilimento A: 849 
Stabilimento B: 0 
Stabilimento C: 0 
Strumento 2 
Stabilimento A: 5093 
Stabilimento B: 1 
Stabilimento C: 0 
Strumento 3 
Stabilimento A: 2082 
Stabilimento B: 2232 
Stabilimento C: 0 
Strumento 4 
Stabilimento A: 0 
Stabilimento B: 0 
Stabilimento C: 0 
Strumento 5 
Stabilimento A: 1 
Stabilimento B: 0 
Stabilimento C: 10912 
Strumento 6 
Stabilimento A: 0 
Stabilimento B: 92 
Stabilimento C: 0 

========== 
ALTRI DATI 
========== 
Ore modifica stabilimento 0: 0.00 
Ore modifica stabilimento 0: 22.50 
Ore modifica stabilimento 0: 0.00 
Numero di adattatori utilizzati: 0 
Numero di moulatori utilizzati: 0 
 
Ricavo massimo azienda = 1801504 
\end{verbatim}
Come si può evincere dai dati, la produzione dello strumento 4 non conviene all'azienda; per essa è preferibile investire il materiale e la manodopera sugli altri strumenti. \\
Durante le prove di esecuzione del programma è emersa una criticità riguardante lo strumento \textit{EDS}: la sua produzione infatti non conviene all'azienda se il prezzo di vendita sta sotto una determinata soglia; superata questa, la produzione diventa redditizia e vantaggiosa. Dal secondo set di dati, infatti, si può evincere che già aumentando di 10\euro  il prezzo di vendita dello strumento in questione i risultati cambiano completamente: da 0 strumenti \textit{EDS} prodotti, infatti, si passa a 919. Aumentando ulteriormente il prezzo di vendita, il risultato dell'esecuzione tende sempre di più a favoreggiare la produzione di tale strumento. Per completezza viene riportato il risultato dell'esecuzione con il secondo set di dati.
\begin{verbatim}

================================ 
VISUALIZZAZIONE PER STABILIMENTI 
================================ 
Stabilimento A 
Strumento 1 : 1
Strumento 2 : 3
Strumento 3 : 6663
Strumento 4 : 0
Strumento 5 : 0
Strumento 6 : 0
Stabilimento B 
Strumento 1 : 0
Strumento 2 : 3
Strumento 3 : 1678
Strumento 4 : 919
Strumento 5 : 0
Strumento 6 : 92
Stabilimento C 
Strumento 1 : 0
Strumento 2 : 0
Strumento 3 : 0
Strumento 4 : 0
Strumento 5 : 10913
Strumento 6 : 0

============================= 
VISUALIZZAZIONE PER STRUMENTI 
============================= 
Strumento 1 
Stabilimento A: 1 
Stabilimento B: 0 
Stabilimento C: 0 
Strumento 2 
Stabilimento A: 3 
Stabilimento B: 3 
Stabilimento C: 0 
Strumento 3 
Stabilimento A: 6663 
Stabilimento B: 1678 
Stabilimento C: 0 
Strumento 4 
Stabilimento A: 0 
Stabilimento B: 919 
Stabilimento C: 0 
Strumento 5 
Stabilimento A: 0 
Stabilimento B: 0 
Stabilimento C: 10913 
Strumento 6 
Stabilimento A: 0 
Stabilimento B: 92 
Stabilimento C: 0 

========== 
ALTRI DATI 
========== 
Ore modifica stabilimento 0: 0.00 
Ore modifica stabilimento 0: 22.50 
Ore modifica stabilimento 0: 0.00 
Numero di adattatori utilizzati: 0 
Numero di moulatori utilizzati: 0 
 
Ricavo massimo azienda = 1803288 
\end{verbatim}