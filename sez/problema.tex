\newpage
\section{Problema}
La ditta C \& B co. produce chitarre e bassi partendo da un set di componenti prestabiliti e assemblandoli insieme per produrre gli strumenti.

Ogni componente esiste in due versioni: per chitarra e per basso; nonostante abbiano lo stesso nome, infatti, non si possono montare i componenti per chitarra su un basso e viceversa, tranne in alcuni casi specificati in seguito. Nella seguente tabella sono indicati nello specifico i componenti, i rispettivi prezzi e, per questioni di mercato, il numero massimo di ognuno che può essere acquistato dall'azienda in un anno.

\begin{table}[htbp]
\resizebox{\textwidth}{!}{
\begin{tabular}{|l|l|l|l|l|}
\hline
                  & \multicolumn{2}{c|}{Chitarra}            & \multicolumn{2}{c|}{Basso}               \\ \hline
                  & \textbf{Quantità acquistabile} & \textbf{Prezzo (\euro /pezzo)} & \textbf{Quantità acquistabile} & \textbf{Prezzo (\euro /pezzo)} \\ \hline
\textbf{Chiavette}         &                       &                  &                       &                  \\ \hline
\textbf{Capotasto}         &                       &                  &                       &                  \\ \hline
\textbf{Manico}            &                       &                  &                       &                  \\ \hline
\textbf{Tastiera}          &                       &                  &                       &                  \\ \hline
\textbf{Tasti}             &                       &                  &                       &                  \\ \hline
\textbf{Segnatasti}        &                       &                  &                       &                  \\ \hline
\textbf{Corpo}             &                       &                  &                       &                  \\ \hline
\textbf{Battipenna}        &                       &                  &                       &                  \\ \hline
\textbf{Ponte}             &                       &                  &                       &                  \\ \hline
\textbf{Pick-up}           &                       &                  &                       &                  \\ \hline
\textbf{Selettore pick-up} &                       &                  &                       &                  \\ \hline
\textbf{Potenziometri}     &                       &                  &                       &                  \\ \hline
\textbf{Jack d'uscita}     &                       &                  &                       &                  \\ \hline
\textbf{Truss-rod}         &                       &                  &                       &                  \\ \hline
\end{tabular}}
\end{table}

La ditta produce quattro modelli di chitarre e due modelli di basso. Ogni modello necessita di una determinata quantità di componenti per poter essere realizzato; i dati sono riassunti nelle seguenti tabelle:
\\ \\
\textbf{Chitarre}
\begin{table}[htbp]
\begin{center}
\resizebox{\textwidth}{!}{
\begin{tabular}{|l|l|l|l|l|l|l|l|l|l|l|l|l|l|l|}
\hline
\textbf{}       & \textbf{Chiav.} & \textbf{Cap.} & \textbf{Man.} & \textbf{Tastiere} & \textbf{Tasti} & \textbf{Segnat.} & \textbf{Corpi} & \textbf{Batt.} & \textbf{Ponti} & \textbf{P.-U.} & \textbf{Selett. P.-U.} & \textbf{Pot.} & \textbf{Jack} & \textbf{T.-R.} \\ \hline
\textbf{LP}     & 6               & 1             & 1             & 1                 & 23             & 9                & 1              & 0              & 2              & 2              & 1                      & 4             & 1             & 1              \\ \hline
\textbf{Strato} & 6               & 1             & 1             & 1                 & 21             & 8                & 1              & 1              & 1              & 3              & 1                      & 3             & 1             & 1              \\ \hline
\textbf{Tele}   & 6               & 1             & 1             & 1                 & 22             & 8                & 1              & 1              & 1              & 1              & 1                      & 2             & 1             & 1              \\ \hline
\textbf{EDS}    & 18              & 2             & 2             & 2                 & 44             & 17               & 2              & 2              & 4              & 4              & 2                      & 4             & 1             & 2              \\ \hline
\end{tabular}}
\end{center}
\end{table}

\textbf{Bassi}

\begin{table}[htbp]
\resizebox{\textwidth}{!}{
\begin{tabular}{|l|l|l|l|l|l|l|l|l|l|l|l|l|l|l|}
\hline
\textbf{}            & \textbf{Chiav.} & \textbf{Cap.} & \textbf{Man.} & \textbf{Tastiere} & \textbf{Tasti} & \textbf{Segnat.} & \textbf{Corpi} & \textbf{Batt.} & \textbf{Ponti} & \textbf{P.-U.} & \textbf{Selett. P.-U.} & \textbf{Pot.} & \textbf{Jack} & \textbf{T.-R.} \\ \hline
\textbf{Mustang}     & 4               & 1             & 1             & 1                 & 19             & 8                & 1              & 1              & 1              & 1              & 1                      & 2             & 1             & 1              \\ \hline
\textbf{Thunderbird} & 5               & 1             & 1             & 1                 & 21             & 9                & 1              & 1              & 1              & 2              & 1                      & 3             & 1             & 1              \\ \hline
\end{tabular}}
\end{table}

Come già detto, esistono delle eccezioni: si può infatti usare un selettore pick-up per chitarra su un basso (e viceversa) aggiungendo un adattatore con costo aggiuntivo di 4\euro , e/o un potenziometro per chitarra su un basso (e viceversa) aggiungendo un modulatore con costo aggiuntivo di 7\euro . \\
\newpage
I prezzi con cui gli strumenti vengono immessi sul mercato sono riportati nella seguente tabella:

\begin{table}[htbp]
\begin{center}
\begin{tabular}{|l|l|}
\hline
\textbf{Strumento} & \textbf{Prezzo di vendita(\euro )} \\ \hline
\textbf{LP}                 &                            \\ \hline
\textbf{Strato}             &                            \\ \hline
\textbf{Tele}               &                            \\ \hline
\textbf{EDS}                &                            \\ \hline
\textbf{Mustang}            &                            \\ \hline
\textbf{Thunderbird}        &                            \\ \hline
\end{tabular}
\end{center}
\end{table}

Per la produzione degli strumenti, l'azienda possiede tre stabilimenti produttivi A, B e C, ognuno con una quantità di ore di manodopera prestabilita; questa quantità è rispettivamente 4000, 5000 e 7000 ore. Ogni strumento può essere prodotto in ognuno dei tre stabilimenti, ma a causa della diversità di mezzi produttivi a disposizione il tempo per produrre un modello in uno stabilimento non è necessariamente lo stesso che si avrebbe in un altro stabilimento; in tabella sono riassunti questi dati.

\begin{table}[htbp]
\begin{center}
\resizebox{\textwidth}{!}{
\begin{tabular}{|l|l|l|l|}
\hline
\textbf{Modello}     & \textbf{Manodopera stab. A} & \textbf{Manodopera stab. B} & \textbf{Manodopera stab. C} \\ \hline
\textbf{LP}          &                             &                             &                             \\ \hline
\textbf{Strato}      &                             &                             &                             \\ \hline
\textbf{Tele}        &                             &                             &                             \\ \hline
\textbf{EDS}         &                             &                             &                             \\ \hline
\textbf{Mustang}     &                             &                             &                             \\ \hline
\textbf{Thunderbird} &                             &                             &                             \\ \hline
\end{tabular}}
\end{center}
\end{table}

Gli strumenti, inoltre, possono essere modificati dalla ditta su richiesta del cliente. Ogni modifica ha lo stesso costo e lo stesso uso di manodopera per ogni modello in ogni stabilimento; il costo per modello, la quantità di manodopera e il ricavo per l'azienda sono di seguito riportati:

\begin{table}[htbp]
\begin{center}
\begin{tabular}{|l|l|l|l|}
\hline
\textbf{Modello}     & \textbf{Costo modifica} & \textbf{Manodopera} & \textbf{Aumento di prezzo} \\ \hline
\textbf{LP}          &                         &                     &					         \\ \hline
\textbf{Strato}      &                         &                     &					         \\ \hline
\textbf{Tele}        &                         &                     &					         \\ \hline
\textbf{EDS}         &                         &                     &					         \\ \hline
\textbf{Mustang}     &                         &                     &					         \\ \hline
\textbf{Thunderbird} &                         &                     &							  \\ \hline     
\end{tabular}
\end{center}
\end{table}

Si richiede di calcolare il mix ottimo di produzione dei vari strumenti, al fine di massimizzare il profitto dell'azienda.