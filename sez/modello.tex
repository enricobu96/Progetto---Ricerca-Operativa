\newpage
\section{Risoluzione del problema}
\subsection{Impostazione del problema}
Prima di procedere con la modellazione e la risoluzione del problema, si è provveduto a definire \textbf{insiemi}, \textbf{parametri} e \textbf{variabili decisionali}.
\paragraph*{Insiemi}
\begin{itemize}
\item[] \textbf{I} = \{1..6\} = tipi di strumenti. Gli indici sono numerici per questioni di leggibilità del modello e del successivo codice \textit{ampl}. Essi corrispondono a:
\begin{enumerate}
\item LP;
\item Strato;
\item Tele;
\item EDS;
\item Mustang;
\item Thunderbird.
\end{enumerate}
\item[] \textbf{J} = \{A, B, C\} = stabilimenti di produzione degli strumenti. 
\item[] \textbf{K} = \{1..14\} = componenti degli strumenti. Anche in questo caso per questioni di leggibilità è stato assegnato a un numero a un componente nel modo seguente:
\begin{enumerate}
\item chiavette;
\item capotasto;
\item manico;
\item tastiera;
\item tasti;
\item segnatasti;
\item corpo;
\item battipenna;
\item ponte;
\item pick-up;
\item selettore pick-up;
\item potenziometri;
\item jack d'uscita;
\item truss-rod.
\end{enumerate}
\item[] \textbf{Tipi} = \{C, B\} = tipologia di strumento, cioè [C]hitarre e [B]assi. Questo insieme non è fondamentale, ma risulta utile per la successiva definizione dei parametri.
\item[] \textbf{Chitarre} = \{1..4\} = sottoinsieme di I, anche questo non indispensabile ma utile per la semantica di parametri e modello del problema.
\item[] \textbf{ComponentiSpeciali} = {11, 12} = sottoinsieme di K, utile (ma non fondamentale) per la modellazione di alcuni vincoli del modello.
\end{itemize}

\paragraph*{Parametri}
\begin{itemize}
\item[] $P_i$ = Prezzo di vendita dello strumento di tipo \textit{i $\in$ I}
\item[] $Y_i$ = Numero di strumento di tipo \textit{i $\in$ I} da modificare; questo viene deciso a priori, quindi non fa parte delle variabili decisionali
\item[] $M_i$ = Guadagno per l'azienda proveniente dalla modifica di uno strumento di tipo \textit{i $\in$ I}
\item[] $CM_i$ = Costo all'azienda per la modifica dello strumento di tipo \textit{i $\in$ I}
\item[] $LS_i$ = Ore di manodopera necessarie per la modifica dello strumento \textit{i $\in$ I}
\item[] $ND_{k,t}$ = Numero di componenti di tipo \textit{k $\in$ K} disponibili per strumento di tipo \textit{t $\in$ Tipi}
\item[] $NN_{k,i}$ = Numero di componenti di tipo \textit{k $\in$ K} necessarie per la costruzione di uno strumento di tipo \textit{i $\in$ I}
\item[] $NP_{k,i}$ = Prezzo del componente \textit{k $\in$ K} per lo strumento \textit{i $\in$ I}. È stato optata questa soluzione (\textit{i $\in$ I} al posto di \textit{t $\in$ Tipi}) per una più semplice modellazione
\item[] $C_i$ = Costo all'azienda dello strumento di tipo \textit{i $\in$ I}; definito come $\sum_{k\in K} NP_{k,i}*NN_{k,i} \forall i\in I$
\item[] $LM_j$ = Numero di ore di manodopera massime dello stabilimento \textit{j $\in$ J}
\item[] $L_{i,j}$ = Numero di ore di manodopera necessarie per produrre uno strumento \textit{i $\in$ I} nello stabilimento \textit{j $\in$ J}
\item[] \textit{Manod} = Costante che indica il costo della manodopera (\euro /ora)
\end{itemize}

\paragraph*{Variabili decisionali}
\begin{itemize}
\item[] $x_{i,j}$ = Numero di strumenti di tipo \textit{i $\in$ I} prodotti nello stabilimento \textit{j $\in$ J}
\item[] y = Numero di adattatori per selettori Pick-Up utilizzati
\item[] z = Numero di modulatori per potenziometri utilizzati
\item[] $v_j$ = Ore di manodopera per la modifica degli strumenti nello stabilimento \textit{j $\in$ J}
\end{itemize}
Oltre a queste, sono state definite altre quattro variabili temporanee con il solo scopo di calcolare y e z mantenendo la linearità del modello; esse sono \textit{a, b, c, d}.

\subsection{Modello matematico}
Essendo questo un problema di tipo \textit{mix ottimo di produzione}, si richiede di massimizzare il guadagno complessivo per l'azienda andando a calcolare la corretta combinazione di tipologie di strumenti da produrre. Il modello è quindi:
\[
\max \overbrace{\sum_{i=1}^{6} \sum_{j\in J} P_i*x_{i,j}}^{Guadagno\ vendita\ strumenti} - \overbrace{\sum_{i=1}^{6} \sum_{j\in J} C_i*x_{i,j}}^{Costo\ produzione\ strumenti} + \overbrace{\sum_{i=1}^{6} M_i*Y_i}^{Guadagno\ modifica\ strumenti} -\]
\[
\overbrace{\sum_{i=1}^{6} CM_i*Y_i}^{Costo\ modifica\ strumenti} - \overbrace{\sum_{i=1}^{6} \sum_{j\in J} (L_{i,j}*x_{i,j})*CostoManod}^{Costo\ modifica\ strumenti} -
\]
\[
\overbrace{\sum_{i=1}^{6} (LS_i*Y_i)*CostoManod}^{Costo\ manodopera\ modifiche} - \overbrace{y - 2z}^{Scambi\ componenti\ chitarra-basso}
\]
\textbf{subject to:} \\
\begin{equation} 
\label{vinc1}
\forall k \in K - ComponentiSpeciali : \left ( \sum_{i\in Chitarre} NN_{k,i}* \sum_{j\in J} x_{i,j} \right) \le ND_{k,C}
\end{equation}

\begin{equation} 
\label{vinc2}
\forall k \in K - ComponentiSpeciali : \left ( \sum_{i\in I-Chitarre} NN_{k,i}* \sum_{j\in J} x_{i,j} \right) \le ND_{k,B}
\end{equation}

\begin{equation} 
\label{vinc3}
\forall k \in ComponentiSpeciali : \left ( \sum_{i\in Chitarre} NN_{11,i}* \sum_{j\in J} x_{i,j} \right) \le a
\end{equation}

\begin{equation} 
\label{vinc4}
\forall k \in ComponentiSpeciali : \left ( \sum_{i\in I-Chitarre} NN_{11,i}* \sum_{j\in J} x_{i,j} \right) \le b
\end{equation}

\begin{equation} 
\label{vinc5}
\left ( a+b \right) \le \sum_{t\in Tipi} ND_{11,t}
\end{equation}

\begin{equation} 
\label{vinc6}
\left( y \right) \ge ND_{11,C}-a;
\end{equation}

\begin{equation} 
\label{vinc7}
\left( y \right) \ge a-ND_{11,C};
\end{equation}

\begin{equation} 
\label{vinc8}
\forall k \in ComponentiSpeciali : \left ( \sum_{i\in Chitarre} NN_{12,i}* \sum_{j\in J} x_{i,j} \right) \le c
\end{equation}

\begin{equation} 
\label{vinc9}
\forall k \in ComponentiSpeciali : \left ( \sum_{i\in I-Chitarre} NN_{12,i}* \sum_{j\in J} x_{i,j} \right) \le d
\end{equation}

\begin{equation} 
\label{vinc10}
\left ( c+d \right) \le \sum_{t\in Tipi} ND_{12,t}
\end{equation}

\begin{equation} 
\label{vinc11}
\left( z \right) \ge ND_{12,C}-c;
\end{equation}

\begin{equation} 
\label{vinc12}
\left( z \right) \ge c-ND_{12,C};
\end{equation}

\begin{equation} 
\label{vinc13}
\forall j \in J : \left ( \sum_{i\in I} (L_{i,j}*x_{i,j}) + v_j \right) \le LM_j
\end{equation}

\begin{equation} 
\label{vinc14}
\left ( \sum_{j\in J} v_j \right) = (\sum_{i\in I} LS_i*Y_i)
\end{equation}


\subsubsection{Spiegazione dei vincoli}
\begin{itemize}
\item[(1)] 
\item[(2)] 
\item[(3)] 
\item[(4)] 
\item[(5)] 
\item[(6)] 
\item[(7)] 
\item[(8)] 
\item[(9)] 
\item[(10)] 
\item[(11)] 
\item[(12)] 
\item[(13)] 
\item[(14)] 
\end{itemize}