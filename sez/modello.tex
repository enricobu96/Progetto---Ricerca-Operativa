\newpage
\section{Risoluzione del problema}
\subsection{Impostazione del problema}
Per risolvere il problema si sono anzitutto definiti gli \textbf{insiemi}; questi sono: 
\begin{itemize}
\item[] \textbf{I} = \{1..6\} = tipi di strumenti. Gli indici sono numerici per questioni di leggibilità del modello e del successivo codice \textit{ampl}. Essi corrispondono a:
\begin{enumerate}
\item LP;
\item Strato;
\item Tele;
\item EDS;
\item Mustang;
\item Thunderbird.
\end{enumerate}
\item[] \textbf{J} = \{A, B, C\} = stabilimenti di produzione degli strumenti. 
\item[] \textbf{K} = \{1..14\} = componenti degli strumenti. Anche in questo caso per questioni di leggibilità è stato assegnato a un numero a un componente nel modo seguente:
\begin{enumerate}
\item chiavette;
\item capotasto;
\item manico;
\item tastiera;
\item tasti;
\item segnatasti;
\item corpo;
\item battipenna;
\item ponte;
\item pick-up;
\item selettore pick-up;
\item potenziometri;
\item jack d'uscita;
\item truss-rod.
\end{enumerate}
\end{itemize}

Successivamente sono stati definiti i \textbf{parametri}, i quali sono:
\begin{itemize}
\item[] $P_i$ = Prezzo di vendita dello strumento di tipo \textit{i $\in$ I}
\item[] $C_i$ = Costo all'azienda dello strumento di tipo \textit{i $\in$ I}
\item[] $Y_i$ = Numero di strumento di tipo \textit{i $\in$ I} da modificare; questo viene deciso a priori, quindi non fa parte delle variabili decisionali
\item[] $M_i$ = Guadagno per l'azienda proveniente dalla modifica di uno strumento di tipo \textit{i $\in$ I}
\item[] $CM_i$ = Costo all'azienda per la modifica dello strumento di tipo \textit{i $\in$ I}
\item[] $ND_{k,T}$ = Numero di componenti di tipo \textit{k $\in$ K} disponibili per strumento di tipo T, con T=C (chitarra) o B (basso)
\item[] $NN_{k,i}$ = Numero di componenti di tipo \textit{k $\in$ K} necessarie per la costruzione di uno strumento di tipo \textit{i $\in$ I}
\item[] $LM_j$ = Numero di ore di manodopera massime dello stabilimento \textit{j $\in$ J}
\item[] $L_{i,j}$ = Numero di ore di manodopera necessarie per produrre uno strumento \textit{i $\in$ I} nello stabilimento \textit{j $\in$ J}
\end{itemize}

Le \textbf{variabili decisionali} sono quindi le seguenti:
\begin{itemize}
\item[] $x_{i,j}$ = Numero di strumenti di tipo \textit{i $\in$ I} prodotti nello stabilimento \textit{j $\in$ J}
\item[] y = Numero di adattatori per selettori Pick-Up utilizzati
\item[] z = Numero di modulatori per potenziometri utilizzati
\end{itemize}

\subsection{Modello matematico}
Essendo questo un problema di tipo \textit{mix ottimo di produzione}, si richiede di massimizzare il guadagno complessivo per l'azienda andando a calcolare la corretta combinazione di tipologie di strumenti da produrre. Il modello è quindi:
\begin{displaymath}
max \overbrace{\sum_{i=1}^{6} \sum_{j\in J} (R_i x_{i,j})}^{Ricavo\ vendita\ strumenti} - \overbrace{\sum_{i=1}^{6} \sum_{j\in J} (C_i x_{i,j})}^{Costo\ strumenti} + \overbrace{\sum_{i=1}^{6} \sum_{j\in J}(G_i x_{i,j})}^{Guadagno\ da\ modifiche} \\
\end{displaymath}
